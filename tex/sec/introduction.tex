\section{Introduction}
\label{sec:introduction}
	
	Through soil analyses or soil testing one can get certain chemical and physical information about the used soil like the concentration of soil organic carbon (SOC) or the $\m{pH}$-value.
	With these Measurements, known as soil parameters, it is possible to optimize plant growth or to assist in solving soil-related problems.
	% [Quelle: https://www.ndsu.edu/soils/services/soil_testing_lab/why_soil_test/]
	
	However the direct measurement of soil parameters is very costly and error-prone.
	Therefore methods for fast and cheap determination of these parameters play a fundamental and vital role in particular fields like agriculture, geochemistry and ecology.
	% [Quelle: https://en.wikipedia.org/wiki/Soil_test]

	At the beginning of the 1960s Karl Norris for the first time in history used Near Infrared Spectroscopy (NIRS) to predict and calculate moisture content from seed extracts through a multivariate calibration approach.
	His work had a huge impact in agricultural an non-agricultural fields since NIRS proved to be a significant time-saver and cheap alternative to other methods.
	% Quelle: Tutorial

	In the last few years NIRS applications experienced a massive growth.
	This would have not been possible without better computing capabilities and progress in multivariate methods as the prediction of soil parameters out of a measured soil spectrum requires a large amount of statistical computations.
	% Quelle: Tutorial

% section introduction