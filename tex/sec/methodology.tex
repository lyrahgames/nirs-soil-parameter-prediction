\section{Methodology}
\label{sec:methodology}
	
	\subsection{Measured Data}
	\label{ssec:measured-data}
	
		As a single spectrum contains overlapping information, it is necessary to determine both relevant wavelengths and the respective parameters to apply NIRS to practical problems.
		To select wavelengths and determine parameters we use an example data set, which contains $p^\m{(SOC)},p^\m{(N)},\m{pH}$ and wave reflectances of 319 wavelengths ranging from $1400 \unit{nm}$ to $2672 \unit{nm}$ by steps of $4 \unit{nm}$ for 533 samples.%citation

		We define $\Lambda$ as the set of all measured wavelengths. The reflectance $\varrho(\lambda)$ of a sample at a wavelength $\lambda \in \Lambda$ is recorded as
		\[
			-\lg \varrho(\lambda) = -\frac{\ln \varrho(\lambda)}{\ln 10}
		\]
		Figure \ref{fig:soil-spec-rnd} shows six randomly chosen soil spectra in a diagram.
		\begin{figure*}
			\centering
			\input{gp/soil-spec-rnd.tex}
			\caption{Six near infrared soil spectra of randomly chosen soil samples obtained from the data set, where $\lambda$ is the wavelength and $\rho(\lambda)$ the corresponding reflectance and each colour refers to one sample}
			\label{fig:soil-spec-rnd}
		\end{figure*}
		
	
	% subsection measured-data

	\subsection{Statistical Model}
	\label{ssec:statistical-model}
	
		Let $n\in\SN$ be the size of the data set and $k\in\SN$ with $k\leq n$ the number of measured wavelengths. We define
		$\varrho_i$ as the soil spectrum of the $i$th sample for every $i\in\SN,i\leq n$.
		$\lambda_j$ is the $j$th measured wavelength for every $j\in\SN,j\leq k$. We will alternatively refer to these as predictors.%note that we might have to introduce the term predictor at this point
		Then according to section \ref{ssec:measured-data} the measured reflectance values are $x_{ij}$ with
		\[
			x_{ij} \define -\lg \varrho_i(\lambda_j)
		\]
		for every $i,j\in\SN,i\leq n,j\leq k$.
		
	
		We define the measured ASF of $\m{SOC}$ of the $i$th sample for every $i\in\SN,i\leq n$ as $p^\m{(SOC)}_i$ to which we will also refer to as response variable.
		To simplify notation, we then define the $n$-dimensional vector
%		\begin{alignat*}{3}
		\[
			p^\m{(SOC)} \define \curvb{p^\m{(SOC)}_i}
		\]
			%p^\m{(N)} &\define&&\ \curvb{p^\m{(N)}_i} \\
			%\m{pH} &\define&&\ \curvb{\m{pH}_i}
%		\end{alignat*}

		The Beer-Lambert law allows us to make assumptions on the relations between soil spectra and the response variable.
		We saw in section \ref{ssec:nirs} that the logarithmised reflectance can be written as a linear combination of molar concentrations.
		Hence, it seems plausible to assume that an ASF can be represented by a linear combination of logarithmised reflectance values.

		Now let $P^\m{(SOC)}$ be the appropriate random vector to $p^\m{(SOC)}$.
		Then under the above assumption the expected values are given for all $i \in \SN, i \leq n$ by
		\[
			\expect P_i^\m{(SOC)} \define \beta^\m{(SOC)}_0 + \sum_{j = 1}^k{x_{ij}\beta^\m{(SOC)}_j} 
		\]
		which simplifies with an $\mathbb{X} \in \SR^{n \times (k+1)}$, called design matrix, and a parameter vector $\beta^\m{(SOC)} \in \SR^{k+1}$ to
		\[
			\expect P^\m{(SOC)} = \mathbb{X}\beta^\m{(SOC)}
		\]
		
		To capture the stochastic part of $P^\m{(SOC)}$, we extend the model to
		\begin{alignat*}{3}
			&P^\m{(SOC)} = \mathbb{X}\beta^\m{(SOC)} + \varepsilon^\m{(SOC)} \\
			&\expect \varepsilon^\m{(SOC)} = 0, \qquad \cov \varepsilon^\m{(SOC)} = (\sigma^2)^\m{(SOC)} \idmat
		\end{alignat*}
		where $(\sigma^2)^\m{(SOC)} \in (0,\infty)$. Following common practice in physics and chemistry, we further assume that 
		\[
			\varepsilon^\m{(SOC)} \sim \FN \curvb{0,(\sigma^2)^\m{(SOC)}\idmat}
		\]
		This results in the complete model
		\[
			P^\m{(SOC)} \sim \FN \curvb{\mathbb{X}\beta^\m{(SOC)},(\sigma^2)^\m{(SOC)} \idmat}
		\]
		The model for the second response variable $P^\m{(N)}$ is constructed in analogy.
		
		The case for the $\m{pH}$ is slightly different, though. 
		When modelling the corresponding random variable we have to adjust the model as the $\m{pH}$ is a logarithmised molar concentration.
		We therefore have to include this into the expected value of the corresponding random variables
		
		\[
			\expect \m{\overline{pH}}_i \define \beta^\m{\m{(pH)}}_0 + \sum_{j = 1}^k{\ln (x_{ij}) \beta^\m{(pH)}_j} 
		\]
		and denote the corresponding matrix by $\mathbb{X}_{\ln}$.

	
	% subsection statistical-model

	\subsection{Multivariate Linear Regression}
	\label{ssec:mlr}
	
		Multiple linear regression (MLR) or multivariate linear regression is a statistical method for estimating parameters of linear relations between a response variable and a set of predictors and to use these to predict new responses.
		Let $\mathbb{X}\in\SR^{n\times(k+1)}, n,k\in\SN,k<n$ be the design matrix, $\sigma^2\in(0,\infty)$ and $Y$ be the random vector variables with
		\[
			Y \sim \FN\curvb{\mathbb{X}\beta,\sigma^2\idmat_n}
		\]
		for a certain $\beta\in\SR^{k+1}$
		Then through the maximum-likelihood-method and a correction we get two best unbiased estimators $\hat{\beta},\hat{\sigma}^2$ for $\beta$ and $\sigma^2$
		\begin{alignat*}{3}
			\hat{\beta}(Y) &=&&\ \inv{\curvb{\transp{\mathbb{X}}\mathbb{X}}}\transp{\mathbb{X}}Y \\
			\hat{\sigma}^2(Y) &=&&\ \frac{1}{n-(k+1)}\norm{Y - \mathbb{X}\hat{\beta}(Y)}^2
		\end{alignat*}
		Now let $y \define (y_i)\in\SR^n$ be a realization of $Y$.
		Then we define
		\begin{alignat*}{3}
			\hat{y} &\define&&\ \mathbb{X}\hat{\beta}(y) = \mathbb{X}\inv{\curvb{\transp{\mathbb{X}}\mathbb{X}}}\transp{\mathbb{X}}y \\
			\hat{\sigma}^2 &\define&&\ \hat{\sigma}^2(y)
		\end{alignat*}
		%citation
	
	% subsection mlr

	\subsection{Mallows' $C_\m{p}$}
	\label{ssec:mallows-C_p}
	
		At this point, the model is specified using $k+1 = 320$ predictors for each response variable, using the whole measured spectra for each soil sample.
		The reflectances of neighbouring lightwaves are correlated. % citation
		This might lead to overfitting, i.e. the variance of our estimated parameters $\hat{\beta}_i(Y)$ might be too large, compromising their usability for future measurements.
		To address this problem, it is sensible to limit each actual model to a \enquote{good} subset of the predictors. Hence, our task becomes to select the best or at least a \enquote{sufficiently} good model $M$ defined by
		\[
			M\subset \Lambda \cup \set{\lambda_0} \definedby \overline{\Lambda}
		\]
		where $\lambda_0$ stands for the intercept.
		We denote the design matrix for each $M$ by $\mathbb{X}^{(M)}$.
		Applying MLR to the new design matrix yields the new estimators
		\begin{alignat*}{3}
			\hat{\beta}^{(M)}(Y) &=&&\ \inv{\curvb{\transp{\mathbb{X}^{(M)}}\mathbb{X}^{(M)}}}\transp{\mathbb{X}^{(M)}}Y \\
			\curvb{\hat{\sigma}^2}^{(M)}(Y) &=&&\ \frac{1}{n-\m{p}}\norm{Y - \mathbb{X}^{(M)}\hat{\beta}^{(M)}(Y)}^2_2,
		\end{alignat*}
		where $\m{p} \in \set{2,\ldots,k}$ corresponds to the number of predictors included in $M$.
		%to follow the logic of our argumentation, we should start with the SPSE
		
		The sum of predicted squared errors ($\m{SPSE}$) is a theoretical criterion to compare the merits of different models.
		The $\m{SPSE}^{(M)}$ measures how well a model does in predicting new responses from new data:
		\[
			\m{SPSE}^{(M)} \define \sum_{i = 1}^n \expect \curvb{Y_{n+i} - \hat{Y}_i^{(M)}}^2
		\]
		which simplifies to % to be elaborated
		\[
			\m{SPSE}^{(M)} = n\sigma^2 + \m{p}\sigma^2
		\]
		Unfortunately, the true $\sigma^2$ is unobservable so that it has to be estimated by
		\[
			\widehat{\m{SPSE}}^{(M)} \define \norm{Y - \mathbb{X}^{(M)}\hat{\beta}^{(M)}(Y)}^2_2 + 2\m{p}\curvb{\hat{\sigma}^2}^{(\overline{\Lambda})}
		\]
		
		Instead of using the $\widehat{\m{SPSE}}^{(M)}$ directly, we will use Mallow's $C_\m{p}$ instead, given by
		\[
			C_\m{p}^{(M)} \define \frac{1}{\curvb{\hat{\sigma}^2}^{(\overline{\Lambda})}}\sum_{i=1}^n\curvb{y_i-\hat{y}^{(M)}_i}^2 - n + 2\m{p}
		\]
		The minimization this value is equivalent to the minimisation of the estimated sum of predicted squared errors ($\widehat{\m{SPSE}}$).
		The proof is trivial, just biject it to a conext-free residue class, whose elements are open manifolds. %citation
	
	% subsection mallows-C_p

	\subsection{Simulated Annealing}
	\label{ssec:model-selec}
	
		As the set of predictors is comparatively large, $n-k < k$, the full-sized model might overfit the data and hence lower the confidence in our parameter estimator.
		The proposed solution in \ref{mallows-C_p} is to reduce the number of predictors.
		Still, with a size of the hypothesis space of $\abs{\s{H}} = 2^{319}$, complete search or even best $k$ approaches are beyond feasibility.
		To reduce the time spent on model search, we will compare two \textsl{do we?} model selection algorithms, simulated annealing and bidirectional elimination.	
		We want to find a subset $M$ such that for all $N\in\s{H}$ the inequality
		\[
			C_\m{p}^{(M)} \leq C_\m{p}^{(N)}
		\]
		holds.

		Simulated annealing (SANN) is a probabilistic technique for approximating the global optimum of a given function.
		Specifically, it is a metaheuristic to approximate global optimisation in large search spaces.
		It simulates the slow cooling of a thermodynamic system through random numbers.
		With this algorithm it is possible to find a \enquote{good} local minimum in a short time.
		% Quelle: https://en.wikipedia.org/wiki/Simulated_annealing

		The algorithm is applicable to arbitrary sets, in our case $\s{H}$.
		Let $x_0\in\s{H}$ be the initial set of predictors, $T_0\in(0,\infty)$ be the initial temperature of the system and $i_\m{max}\in\SN$ be the maximal number of time steps.
		Then the algorithm requires the following functions:
		\begin{itemize}
			\item $\func{\m{cost}}{\s{H}}{\SR}$ \\
				Calculates the cost of a given predictor set.
			\item $\func{\m{temp}}{\SR\times\SN^2}{(0,\infty)}$\\
				Calculates the temperature according to the given initial temperature and time steps.
				It is a monotonically decreasing function in the second parameter.
			\item $\func{\m{nbr}}{\s{H}}{\s{H}}$ \\
				Generates a random neighbor of a given predictor set.
			\item $\func{\m{prob}}{\SR^2\times(0,\infty)}{[0,1]}$ \\
				Calculates the probability of changing the current set or state to the neighbor.
			\item $\m{rnd}(0,1)$ \\
				Returns a random number in the interval $[0,1]$.
		\end{itemize}
		The listing shows one variant of the pseudocode of the SANN algorithm. 

		\medskip
		\begin{tcolorbox}[colframe=black,colbacktitle=white,coltitle=black, attach boxed title to top center={yshift=-2mm},enhanced, titlerule=0.1pt, boxrule=0.5pt, arc=5pt,title=Listing:\quad SANN algorithm]
			\begin{tabbing}
	\qquad\=\qquad\=\qquad\=\qquad\=\kill
	$c_0 = \m{cost}(x_0)$\\
	\\
	\textbf{for} ($i=1$, $i\leq i_\m{max}$) \{\\
		\>$T = \m{temp}(T_0,i,i_\m{max})$\\
		\\
		\>$x_1 = \m{nbr}(x_0)$\\
		\>$c_1 = \m{cost}(x_1)$\\
		\\
		\>\textbf{if} $(\m{prob}(c_0, c_1, T) \geq \m{rnd}(0,1))$ \{\\
			\>\>$x_0 = x_1$\\
			\>\>$c_0 = c_1$\\
		\>\}\\
	\}
\end{tabbing}
		\end{tcolorbox}
		\medskip

	% subsection simulated-annealing
		
	\subsection{Model Validation}
	\label{ssec:model-validation}
	
		% goodness of prediction/R^2 discussion
		To examine the models selected through \ref{ssec:model-selec}, we recur to the often used $R^2$ measure.
		It is given by
		\[
			\curvb{R^2}^{(M)} \define \frac {\sum_{i=1}^n \curvb{\hat{y}_i - \overline{y}}^2}{\sum_{i=1}^n \curvb{y_i - \overline{y}}^2}
		\]
		and describes how much of the total variation of $y$ is explained by the selected model.
		
		In addition, we will use correlation diagrams for each model, plotting $\hat{y}$ on $y$, to check for grave divergences.
	% subsection model-validation
	
	\subsection{Simulation}
	\label{ssec:simulation}
	
		Having chosen the selection criterion based on the performance in predicting the responses, we need to assess how well the model selection algorithm minimises the $\m{SPSE}$.
		We have seen in \ref{ssec:mallows-C_p}, our algorithm heuristically minimises the estimator $\widehat{\m{SPSE}}$.
		
		As the true $\m{SPSE}$ is unobservable, we resort to simulations of one response variable $\tilde{Y}^{(j)}, j \in \SN, j \leq 1000$, where we can fix both, $\expect{\tilde{Y}^{(j)}}$ and $C^{(j)}$.
		The resulting values for the response variable are also called pseudo-observations.
		We then apply our model selection algorithm to each simulation and compare the resulting estimators $\widehat{\m{SPSE}}^{(j)}$ with the \enquote{true} $\m{SPSE}$ computed from the fixed parameters.
				
		The pseudo-observations of one simulation are generated as follows
		\begin{alignat*}{3}
			\tilde{Y} &&\, = \, & \hat{Y} + \varepsilon \\
			\varepsilon &&\, \sim \, & \FN \curvb{0,\hat{\sigma}^2\idmat}
		\end{alignat*}
		where $\hat{Y}$ is the selected model from \ref{ssec:model-selec} and $\hat{\sigma}^2$ the estimator for $\sigma^2$ in said model.
		We then proceed to use the complete model space $\overline{\Lambda}$ to select the best model for each simulation and estimate the $\m{SPSE}$ through $\widehat{\m{SPSE}}^{(j)}$.
		We then compare the estimates with the fixed \enquote{true} $\m{SPSE}$.
		
		To gauge how much the resolution of the NIRS influences the model selection, we repeat the model selection on the simulations, but allow only models from
		\[
			\Lambda^{m} \define \set{\lambda_{mi} | \lambda_{mi} \in \Lambda, i = 1, 2, \ldots} \qquad m \in \set{2, 3, 4}
		\]
		where $\overline{\Lambda}^{m}$ takes the place of $\overline\Lambda$ in the algorithm.
		
		Just as with resolution, we can also assess the influence of availability of measurements.
		Choosing a suitable $N \in \SN^L$ with entries $N_l \ge k, l \in \set{1,\ldots,L}$, we randomly select a subset of the simulation data of size $N_l$ and proceed as above.
		As $n-k \le k$, we perform this inquiry with $\Lambda^{3}$ instead to reduce computational cost and capture a larger variety of L.
		
		Using the same algorithm for all three response variables, we can limit the algorithm examination to one response variable, $\m{SOC}$, to be better able to compare the results and reduce redundant analyses.		
		
	% subsection simulation
% section methodology